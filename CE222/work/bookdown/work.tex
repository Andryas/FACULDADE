\documentclass[]{article}
\usepackage{lmodern}
\usepackage{amssymb,amsmath}
\usepackage{ifxetex,ifluatex}
\usepackage{fixltx2e} % provides \textsubscript
\ifnum 0\ifxetex 1\fi\ifluatex 1\fi=0 % if pdftex
  \usepackage[T1]{fontenc}
  \usepackage[utf8]{inputenc}
\else % if luatex or xelatex
  \ifxetex
    \usepackage{mathspec}
  \else
    \usepackage{fontspec}
  \fi
  \defaultfontfeatures{Ligatures=TeX,Scale=MatchLowercase}
\fi
% use upquote if available, for straight quotes in verbatim environments
\IfFileExists{upquote.sty}{\usepackage{upquote}}{}
% use microtype if available
\IfFileExists{microtype.sty}{%
\usepackage{microtype}
\UseMicrotypeSet[protrusion]{basicmath} % disable protrusion for tt fonts
}{}
\usepackage[margin=1in]{geometry}
\usepackage{hyperref}
\hypersetup{unicode=true,
            pdfborder={0 0 0},
            breaklinks=true}
\urlstyle{same}  % don't use monospace font for urls
\usepackage{natbib}
\bibliographystyle{apalike}
\usepackage{color}
\usepackage{fancyvrb}
\newcommand{\VerbBar}{|}
\newcommand{\VERB}{\Verb[commandchars=\\\{\}]}
\DefineVerbatimEnvironment{Highlighting}{Verbatim}{commandchars=\\\{\}}
% Add ',fontsize=\small' for more characters per line
\usepackage{framed}
\definecolor{shadecolor}{RGB}{248,248,248}
\newenvironment{Shaded}{\begin{snugshade}}{\end{snugshade}}
\newcommand{\KeywordTok}[1]{\textcolor[rgb]{0.13,0.29,0.53}{\textbf{#1}}}
\newcommand{\DataTypeTok}[1]{\textcolor[rgb]{0.13,0.29,0.53}{#1}}
\newcommand{\DecValTok}[1]{\textcolor[rgb]{0.00,0.00,0.81}{#1}}
\newcommand{\BaseNTok}[1]{\textcolor[rgb]{0.00,0.00,0.81}{#1}}
\newcommand{\FloatTok}[1]{\textcolor[rgb]{0.00,0.00,0.81}{#1}}
\newcommand{\ConstantTok}[1]{\textcolor[rgb]{0.00,0.00,0.00}{#1}}
\newcommand{\CharTok}[1]{\textcolor[rgb]{0.31,0.60,0.02}{#1}}
\newcommand{\SpecialCharTok}[1]{\textcolor[rgb]{0.00,0.00,0.00}{#1}}
\newcommand{\StringTok}[1]{\textcolor[rgb]{0.31,0.60,0.02}{#1}}
\newcommand{\VerbatimStringTok}[1]{\textcolor[rgb]{0.31,0.60,0.02}{#1}}
\newcommand{\SpecialStringTok}[1]{\textcolor[rgb]{0.31,0.60,0.02}{#1}}
\newcommand{\ImportTok}[1]{#1}
\newcommand{\CommentTok}[1]{\textcolor[rgb]{0.56,0.35,0.01}{\textit{#1}}}
\newcommand{\DocumentationTok}[1]{\textcolor[rgb]{0.56,0.35,0.01}{\textbf{\textit{#1}}}}
\newcommand{\AnnotationTok}[1]{\textcolor[rgb]{0.56,0.35,0.01}{\textbf{\textit{#1}}}}
\newcommand{\CommentVarTok}[1]{\textcolor[rgb]{0.56,0.35,0.01}{\textbf{\textit{#1}}}}
\newcommand{\OtherTok}[1]{\textcolor[rgb]{0.56,0.35,0.01}{#1}}
\newcommand{\FunctionTok}[1]{\textcolor[rgb]{0.00,0.00,0.00}{#1}}
\newcommand{\VariableTok}[1]{\textcolor[rgb]{0.00,0.00,0.00}{#1}}
\newcommand{\ControlFlowTok}[1]{\textcolor[rgb]{0.13,0.29,0.53}{\textbf{#1}}}
\newcommand{\OperatorTok}[1]{\textcolor[rgb]{0.81,0.36,0.00}{\textbf{#1}}}
\newcommand{\BuiltInTok}[1]{#1}
\newcommand{\ExtensionTok}[1]{#1}
\newcommand{\PreprocessorTok}[1]{\textcolor[rgb]{0.56,0.35,0.01}{\textit{#1}}}
\newcommand{\AttributeTok}[1]{\textcolor[rgb]{0.77,0.63,0.00}{#1}}
\newcommand{\RegionMarkerTok}[1]{#1}
\newcommand{\InformationTok}[1]{\textcolor[rgb]{0.56,0.35,0.01}{\textbf{\textit{#1}}}}
\newcommand{\WarningTok}[1]{\textcolor[rgb]{0.56,0.35,0.01}{\textbf{\textit{#1}}}}
\newcommand{\AlertTok}[1]{\textcolor[rgb]{0.94,0.16,0.16}{#1}}
\newcommand{\ErrorTok}[1]{\textcolor[rgb]{0.64,0.00,0.00}{\textbf{#1}}}
\newcommand{\NormalTok}[1]{#1}
\usepackage{longtable,booktabs}
\usepackage{graphicx,grffile}
\makeatletter
\def\maxwidth{\ifdim\Gin@nat@width>\linewidth\linewidth\else\Gin@nat@width\fi}
\def\maxheight{\ifdim\Gin@nat@height>\textheight\textheight\else\Gin@nat@height\fi}
\makeatother
% Scale images if necessary, so that they will not overflow the page
% margins by default, and it is still possible to overwrite the defaults
% using explicit options in \includegraphics[width, height, ...]{}
\setkeys{Gin}{width=\maxwidth,height=\maxheight,keepaspectratio}
\IfFileExists{parskip.sty}{%
\usepackage{parskip}
}{% else
\setlength{\parindent}{0pt}
\setlength{\parskip}{6pt plus 2pt minus 1pt}
}
\setlength{\emergencystretch}{3em}  % prevent overfull lines
\providecommand{\tightlist}{%
  \setlength{\itemsep}{0pt}\setlength{\parskip}{0pt}}
\setcounter{secnumdepth}{5}
% Redefines (sub)paragraphs to behave more like sections
\ifx\paragraph\undefined\else
\let\oldparagraph\paragraph
\renewcommand{\paragraph}[1]{\oldparagraph{#1}\mbox{}}
\fi
\ifx\subparagraph\undefined\else
\let\oldsubparagraph\subparagraph
\renewcommand{\subparagraph}[1]{\oldsubparagraph{#1}\mbox{}}
\fi

%%% Use protect on footnotes to avoid problems with footnotes in titles
\let\rmarkdownfootnote\footnote%
\def\footnote{\protect\rmarkdownfootnote}

%%% Change title format to be more compact
\usepackage{titling}

% Create subtitle command for use in maketitle
\newcommand{\subtitle}[1]{
  \posttitle{
    \begin{center}\large#1\end{center}
    }
}

\setlength{\droptitle}{-2em}
  \title{}
  \pretitle{\vspace{\droptitle}}
  \posttitle{}
  \author{}
  \preauthor{}\postauthor{}
  \date{}
  \predate{}\postdate{}

\usepackage{booktabs}
% \usepackage{xcolor}
\usepackage{amsthm}
\usepackage{float}
\usepackage[utf8]{inputenc}
\usepackage[brazil]{babel}
\usepackage[table]{xcolor}
\usepackage{eso-pic}
\newcommand\BackgroundPic{%
\put(0,0){%
\parbox[b][\paperheight]{\paperwidth}{%
\vfill
\centering
\includegraphics[width=\paperwidth,height=\paperheight,%
keepaspectratio]{ufpr.jpg}%
\vfill
}}}
\AddToShipoutPicture*{\BackgroundPic}
\usepackage{booktabs}
\usepackage{longtable}
\usepackage{array}
\usepackage{multirow}
\usepackage[table]{xcolor}
\usepackage{wrapfig}
\usepackage{float}
\usepackage{colortbl}
\usepackage{pdflscape}
\usepackage{tabu}
\usepackage{threeparttable}
\usepackage[normalem]{ulem}

\begin{document}

\begin{titlepage}
\centering{\large{UNIVERSIDADE FEDERAL DO PARANÁ}}
\\
\centering{Departamento de Estatística}

\vspace{3.5cm}

\centering{\huge{TITULO}}

\vspace{3cm}

\centering{\Large{Prof. Benito Olivares Aguilera}}

\vspace{3cm}

CE222 - Processos Estocásticos Aplicados

\vspace{2cm}

Andryas Waurzenczak, GRR: 20149125 \\
Caleb Josué Souza, GRR: 20149072



\vfill

13/06/2018
\end{titlepage}


\begin{abstract}
  Este estudo tem como objetivo apresentar uma alternativa para a estimação
  do valor da volatilidade utilizando os conhecimentos probabilísticos
  envolvidos em um  processo estocástico. Entende-se por volatilidade, no
  universo financeiro, como uma variável ou um elemento que está presente em
  uma métrica de risco de mercado. A variável volatilidade indica a
  intensidade e a frequência das oscilações de preços de ações em um
  determinado período de tempo. Serão apresentadas duas maneiras para estimar
  a volatilidade, sendo a primeira o método de estimação pelo desvio padrão
  histórico e a segunda o método conhecido como Volatilidade
  Estocástica. Juntamente com explanação teórica envolvida em cada um destes
  dois métodos também serão apresentados os resultados comparativos
  adquiridos através da análise feita em cima de um histórico das ações
  BBDC4.SA.

  Palavras Chave: VaR, Volatilidade Estocástica
\end{abstract}


\pagebreak
\tableofcontents
\pagebreak

\section{Introdução}\label{introducao}

O mercado de ações é um ambiente público e organizado para negociação de
alguns títulodes mobiliários (ações, opções de fundos, etc\ldots{}).
Neste trabalho o objeto de estudo são títulodes mobiliários chamados de
ações.

Ação é a menor parcela do capital social das companhias de capital
aberto, que têm seus papéis negociados na \href{}{BM\& Bovespa} (Bolsa
de Valores do Brasil). É um título patrimonial e concede aos seus
dententores, chamados de acionistas, os direitos e deveres de um sócio.

Estas ações são negociadas diariamente e com o avanço da tecnologia elas
se tornam cada vez mais rápidas, isso faz com que o mercado se torne
cada vez mais complexo. O principal objetivo de se entrar nesse mundo de
acionistas é o ganho de dinheiro, que se torna cada vez mais difícil
pelo motivo citado anteriormente e pela competitividade mais acirrada.
Assim estratégias e medidas de risco podem auxiliar o acionista a um
melhor desempenho em suas negociações.

\begin{center}\rule{0.5\linewidth}{\linethickness}\end{center}

A volatilidade é um termo bem conhecido àquelas pessoas que estão
envolvidas direta ou indiretamente no mercado financeiro, seja um
operador experiente ou um então um investidor. Saber manejar os
resultados das estimativas desta variável é extremamente importante,
pois isto implica diretamente nos resultados e retornos de
investimentos. Quando se fala em investimento o termo risco é o grande
ditador do potencial de perda ou ganho de dinheiro, quanto maior for o
risco de um investimento maior será também os ganhos e em contrapartida
maior será também as perdas. A comparação inversa para o menor risco
seguirá a mesma lógica. O conceito de volatilidade está diretamente
ligado à métrica do risco, pois esta variável indica a intensidade e a
frequência das oscilações de preços de ativos, mostra o quanto o valor
de um ativo se desvia do seu preço médio ou do seu preço atual.

Existem vários métodos para estimativa da volatilidade, entre eles está
o método EWMA, que é chamada como média móvel exponencial ponderada, que
considera as observações históricas diárias e pondera os resultados mais
recentes com um peso maior. Existem também os modelos ARCH
(Autoregressive Conditional) e sua extenção GARCH que é a generalização
do ARCH que são modelos de séries temporais. Além destes existem ainda o
método de desvio padrão e o modelo Volatilidade estocástica. Estes dois
últimos são os objetos deste estudo.

Este artigo está dividido em 5 seções sendo que a seção dois será
apresentado o método de estimativa por desvio padrão, na seção três será
apresentado o modelo de volatilidade estocátistca e em seguida na seção
quatro será exposto os resultados provenientes da análise comparativa
entre os modelos estudados, e por fim na quinta e ultima seção será
apresentado a conclusão.

\section{Materiais e Métodos}\label{materiais-e-metodos}

\subsection{Materias}\label{materias}

O material utilizado é um conjunto de dados, das negociações da empresa
BBDC4.SA, retirado do \href{https://finance.yahoo.com/}{Yahoo Finance!}.
A empresa foi selecionada pela preferência dos autores. O período que
abragem esta seleção vai de 04/01/2016 à 30/05/2018. Os dados são
diários.

O conjunto de dados possuem ao todo 6 colunas, sendo estas, \emph{Data},
\emph{Máxima}, \emph{Miníma}, \emph{Abertura}, \emph{Fechamento} e
\emph{Volume}. Abaixo pela Tabela \ref{tab:head} pode-se observar as 6
primeiras linhas dos dados.

\rowcolors{2}{gray!6}{white}

\begin{table}[!h]

\caption{\label{tab:head}6 primeiras linhas dos conjuntos de dados}
\centering
\begin{tabular}[t]{lrrrrr}
\hiderowcolors
\toprule
Data & Abertura & Fechamento & Máxima & Miníma & Volume\\
\midrule
\showrowcolors
2015-11-23 & 17.8062 & 17.9113 & 17.2051 & 17.4080 & 20650864\\
2015-11-24 & 17.2051 & 17.3854 & 16.8670 & 17.2878 & 14357497\\
2015-11-25 & 16.9722 & 17.0398 & 16.4162 & 16.4463 & 22973992\\
2015-11-26 & 16.4763 & 16.9121 & 16.4538 & 16.8069 & 14635676\\
2015-11-27 & 16.6642 & 16.6867 & 16.1232 & 16.1908 & 18570778\\
2015-11-30 & 16.0030 & 16.1683 & 15.6649 & 15.8678 & 43981963\\
\bottomrule
\end{tabular}
\end{table}

\rowcolors{2}{white}{white}

\subsection{Métodos}\label{metodos}

\subsubsection{Retorno Contínuo}\label{retorno-continuo}

O retorno utilizado é o retorno contínuo diário dado da seguinte forma:

\[y_t = ln(\frac{P_t}{P_{t-1}}), \text{ t = 1, 2, ..., T}\]

onde \(y_t\) é o retorno diário no instante t e \(P_t\) significa,
normalmente, o Preço de \textbf{fechamento} no instante t.

\begin{Shaded}
\begin{Highlighting}[]
\CommentTok{# r: Preço de Fechamento, Abertura, Máxima ou Miníma}
\NormalTok{rc <-}\StringTok{ }\ControlFlowTok{function}\NormalTok{(r) \{}
    \CommentTok{# Instante inicial}
\NormalTok{    rc <-}\StringTok{ }\KeywordTok{c}\NormalTok{(}\DecValTok{0}\NormalTok{)}
    \ControlFlowTok{for}\NormalTok{ (i }\ControlFlowTok{in} \DecValTok{2}\OperatorTok{:}\KeywordTok{length}\NormalTok{(r)) \{}
\NormalTok{        rc <-}\StringTok{ }\KeywordTok{c}\NormalTok{(rc, }\KeywordTok{log}\NormalTok{(r[i] }\OperatorTok{/}\StringTok{ }\NormalTok{r[i }\OperatorTok{-}\StringTok{ }\DecValTok{1}\NormalTok{]))}
\NormalTok{    \}}
\NormalTok{    rc}
\NormalTok{\}}
\end{Highlighting}
\end{Shaded}

\subsubsection{Volatilidade Simples}\label{volatilidade-simples}

A estimativa por desvio padrão é considerado como um método simplório,
pois este método ignora caracaterísticas inerentes de uma série além de
atribuir pesos iguais a todas observações sem levar em consideração o
fato da relevancia das observações mais recentes. É expressa da seguinte
maneira:

\[\sigma_s = \sqrt(var(y_t)), \text{ t = 1, 2, ..., T}\]

para cada dia a volatilidade deste período é a mesma.

\begin{Shaded}
\begin{Highlighting}[]
\CommentTok{# r: Preço de Fechamento, Abertura, Máxima ou Miníma}
\CommentTok{# n: Número de instantes considerados}
\NormalTok{vs <-}\StringTok{ }\ControlFlowTok{function}\NormalTok{(r, n) \{}
\NormalTok{    vol <-}\StringTok{ }\KeywordTok{rep}\NormalTok{(}\DecValTok{0}\NormalTok{, n)}
    \ControlFlowTok{for}\NormalTok{ (i }\ControlFlowTok{in}\NormalTok{ (n }\OperatorTok{+}\StringTok{ }\DecValTok{1}\NormalTok{)}\OperatorTok{:}\KeywordTok{length}\NormalTok{(r)) \{}
        \CommentTok{# Volatilidade no Instante t}
\NormalTok{        vol <-}\StringTok{ }\KeywordTok{c}\NormalTok{(vol, }\KeywordTok{sd}\NormalTok{(r[(i }\OperatorTok{-}\StringTok{ }\NormalTok{n)}\OperatorTok{:}\NormalTok{(i }\OperatorTok{-}\StringTok{ }\DecValTok{1}\NormalTok{)]))}
\NormalTok{    \}}
\NormalTok{    vol}
\NormalTok{\}}
\end{Highlighting}
\end{Shaded}

\subsubsection{Volatilidade Estocástica}\label{volatilidade-estocastica}

\subsubsection{Valor em Risco (VaR)}\label{valor-em-risco-var}

\section{Resultados}\label{resultados}

\subsection{Análise Descritiva}\label{analise-descritiva}

Para dar inicio aos resultados é usual explorar algumas medidas
descritivas. Primeiramente examina-se a variação dos preços no período
em estudo pelo gráfico de candles abaixo.

\begin{center}\includegraphics{work_files/figure-latex/candles-1} \end{center}

Pelo gráfico acima observa-se uma valorização da empresa no período
selecionado.

\subsection{Retorno Contínuo}\label{retorno-continuo-1}

Para a aplicação de modelos de volatilidade para o cálculo do valor de
risco (VaR) é usual o uso do retorno contínuo descrito na seção de
métodos. Abaixo temos o gráfico dos log-retornos.

\begin{center}\includegraphics{work_files/figure-latex/log-retorno-1} \end{center}

\subsection{Volatilidade Simples}\label{volatilidade-simples-1}

\begin{center}\includegraphics{work_files/figure-latex/volatildiade-est-simples-1} \end{center}

\subsection{Modelo de Volatilidade
Estocástica}\label{modelo-de-volatilidade-estocastica}

\section{Considerações Finais}\label{consideracoes-finais}

\section{Anexo}\label{Anexo}

\bibliography{book.bib}


\end{document}
