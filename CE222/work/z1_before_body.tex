\begin{titlepage}
\centering{\large{UNIVERSIDADE FEDERAL DO PARANÁ}}
\\
\centering{Departamento de Estatística}

\vspace{3.5cm}

\centering{\huge{Modelo de Volatilidade Estocástica}}

\vspace{3cm}

\centering{\Large{Prof. Benito Olivares Aguilera}}

\vspace{3cm}

CE222 - Processos Estocásticos Aplicados

\vspace{2cm}

Andryas Waurzenczak, GRR: 20149125 \\
Caleb Josué Souza, GRR: 20149072



\vfill

13/06/2018
\end{titlepage}


\begin{abstract}
  Este estudo tem como objetivo apresentar uma alternativa para a estimação
  do valor da volatilidade utilizando os conhecimentos probabilísticos
  envolvidos em um  processo estocástico. Entende-se por volatilidade, no
  universo financeiro, como uma variável ou um elemento que está presente em
  uma métrica de risco de mercado. A variável volatilidade indica a
  intensidade e a frequência das oscilações de preços de ações em um
  determinado período de tempo. Serão apresentadas duas maneiras para estimar
  a volatilidade, sendo a primeira o método de estimação pelo desvio padrão
  histórico e a segunda o método conhecido como Volatilidade
  Estocástica. Juntamente com explanação teórica envolvida em cada um destes
  dois métodos também serão apresentados os resultados comparativos
  adquiridos através da análise feita em cima de um histórico das ações
  BBDC4.SA.
  \\
  \\
  Palavras Chave: VaR, Volatilidade Estocástica
\end{abstract}


\pagebreak
\tableofcontents
\pagebreak